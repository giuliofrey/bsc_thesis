\documentclass[11pt, titlepage]{article}
%\usepackage[fontsize=12pt]{scrextend}


%TItle Author And Date
\author{Giulio Frey}
\title{Understanding accessibility to basic services for public housing in Milan: An open source approach}
\date{\textit{Last update: \today}}
%packeges
\usepackage{amsmath}
\usepackage{amsthm}
\usepackage{amsfonts}
\usepackage{multicol}
\usepackage[utf8]{inputenc}
\usepackage{csquotes}
\usepackage{geometry}
\usepackage{ragged2e}
\usepackage[english]{babel}
\usepackage{tabularx}
\usepackage{graphicx}
\usepackage{booktabs}
\usepackage{lscape}
\usepackage{pgfplots}
\usepackage{tkz-fct}
\usepackage[useregional]{datetime2}
\usepackage[style=authoryear, backend=biber]{biblatex}
%interlinea  
\usepackage{setspace}
\onehalfspacing
%\doublespacing
%Imports biblatex package
\addbibresource{bibliography.bib} 
%Page settings
\geometry{
	a4paper,
	total={170mm,257mm},
	left=20mm,
	top=20mm,
}
\setlength\parindent{0pt}
\justifying

\setlength{\parindent}{0em}
\setlength{\parskip}{0.5em}

%commands

\newcommand{\est}{\hat{\theta}_n}

\newtheorem*{theorem}{Theorem}

\newtheorem{innercustomthm}{Theorem}
\newenvironment{customthm}[1]
{\renewcommand\theinnercustomthm{#1}\innercustomthm}
{\endinnercustomthm}

\pgfmathdeclarefunction{p}{3}{%
	\pgfmathparse{(and(#1>#2, #1<#3))}%
}

\newcommand{\Tolerance}{0.0001}%
\pgfmathdeclarefunction{f}{1}{%
	\pgfmathparse{%
		p(#1,0,1-\Tolerance)*1.0+%
		p(#1,1,\maxdimen)*(1/(x))}%
}



%graphics path

\graphicspath{{C:/Users/iodio/Projects/mm/outputs/plots/}}

\begin{document}
		\maketitle


\section{Introduction}


One house every ten in Milan is public  . Public housing is not only highly widespread but also requested:  just 31\% of the target population is assigned to public housing \parencite{comune2023}. Although Milan housing market is classified as fairly valued by \cite{ubsbubble}, a rapid change of the cities population is happening: in the last decade 40\% of the city population was replaced. The economic attractiveness of the city may push out some of the most vulnerable parts of the population, that on the other hand may be the ones that necessitate the most form opportunities of the city \parencite{Chetty2017}.

Public housing is thus a necessary policy tool to allow lower income families to fully interact with the economic benefits coming from living in an highly productive environment. Efficient allocation of public housing is thus necessary, but in the city of milan this does not always happen. The Public Housing stock, in fact, is not fully assigned. Many tenants, after the assignation of an apartment stay there for the rest of their life and thus when the apartment is freed (after 20-30 years) it is in need of restorations \parencite{maran2023}. As resources are limited, it is not easy to prioritize which should be the areas where the public investment for renovation of the house has an higher return for the tenants.

We propose a tool for the evaluation of connectivity of the different social housing buildings with necessary services around the city. The aim is to add another source of analysis in the decision-making processes of areas  in need of renovation. We will highting both those areas that are highly connected on services and thus most suitable for public housing development and renovation and also compare existing public housing stock to understand were services are missing.  The approach is fully open source, thus it has very limited cost for implementation. It is also scalable and reproducible for other cities and uses.

ADD SMALL SUMMARY OF RESULTS 

This paper is structured as follows: section \ref{lirev} provides an overview of literature on urban analytics. Section \ref{back} covers a comprehensive history of public housing institutions and buildings in Milan and a synthetic description of the public housing stock owned by Comune di Milano. Section \ref{data} explains data source and how the distance matrix is calculated using the Open Source Routing Machine. Section \ref{edesign} goes over the design of the connectivity index used for the analysis. Section \ref{res} explains the results obtained and provides some insights on the metric used for the analysis of the connectivity index. Section \ref{conclusion} draws conclusion and appendix \ref{maps} provides maps created using the tools described in this paper.

\section{Literature Review}
\label{lirev}

Data science's tools are increasingly applied to urban planning, soo much that this process earned its own name in literature: urban analytics. Those tools see applications in many fields related to urban planning: from urban transports to energy, air quality, health, policing and transport management \parencite{Kandt2021}. 

Multiple authors showed how spacial accessibility is an important factor to be considered in the policy making process. \cite{Sharma2022} show how unequal supply of educational opportunities across the city can further aggravate social inequality between neighborhoods in Mumbai, India. \cite{Yin2018} show how variable is China's spacial accessibility to healthcare and how this is correlated with socio-economical characteristics of the population. In \cite{Grubesic2017} spacial accessibility to breastfeeding support is used to assess policies addressing this issue in Ohio, USA

Those studies are only focused on one specific service and its easiness of access without considering an overall basket of services that any individual needs. According to \cite{wang2005assessing}
focusing on only one type of service may not consider the substitution effect of public services facilities and how the lack of easy access to one service may be compensated by another one.

Other issues of current spacial analysis research are highted by \cite{Li2021}. Firstly previous research has been focused on car as a standard mode of transportation, that does not apply correctly to dense urban environments and does not consider disadvantaged groups that do not have access to private mobility. Secondly, authors have tended to use threshold distances and district level analysis rather than individual based travel times that may better represent accessibility.

Our approach fully addresses all this issues as it is multiservice, individual based and our index considers pedestrian travel time in parallel with variability of choice of the same service.
 
\section{Background}
\label{back}

Italy begun the development of national public housing in 1903 with Legge Luzzati, before public intervention on the market for housing was almost inexistent. This law was the direct consequence of two main factors affecting Italy at the turn of the century: first a sociological change started in 1898 by popular uprisings all across the nation were the conditions of the most vulnerable were increasingly evident. Secondly the installment of the second Giolitti term that moved italian politics towards reformism and socialism.  The aim of this law was to create better housing condition for the increasing amount of blue collar workers moving to italian cities \parencite{istitutocasepopolari}. In this legal framework the institutions responsible fo public housing were not directly responsible for development but  were only regulators of privates entities \parencite{urbani}. In the 1909 the City of Milan creates the Istituto Autonomo per Le Case Popolari (Autonomous institute for public housing), this institute will evolve WERE IT EVOLVS


VUOTO STORICO 1900-1950

Only during the 1950s the State assumes a more relevant but still indirect position by giving additional responsibility to  the national insurance program to provide public housing. In between 1978 and 1988, in order to meet the increase in demand for housing that drove upwards prices, 2000 housing units per year were added in all lombardy. Following this period of expansion the supply of public housing cooled down to around 700 of new housing units added per year in the period of 1989 and 1995 and finally to reach 450 housing units per year in the period 1995-2001 \parencite{edilizaaler}. This reduction was mainly caused by two factors: a reduction of public funds and law 560 (``Norme in materia di alienazione degli alloggi ERP"), that asked for a reduction of the public housing stock of 50\% in order to fund the renovation of the existing stock. Only one third of the sold housing units was replaced by new constructions, far from a one to one replacement as initially planned  \parencite{Cognetti}.

Since december 2014 all the buildings owned by Comune di Milano are maintained by MM s.p.a. replacing ALER that previously had this responsibility.

\cite{breda_tua_2016} and \cite{Breda} provide a thorough description of the the housing stock owned by Comune di Milano. We will cover the main insights. 

The housing stock owned by Comune di Milano is spread all around the city. Although the acquisition of public housing started in the twentieth century, some building were constructed before and later acquired. For example via Bergamini 1 was build before the eitheen century and acquired by Comune di Milano in 1938. This causes some of the units to be located very close to the city center, in locations with high property value. 

The first public housing project specifically constructed for this use were the two build by Società Umanitaria. Those projects are located in Via Andrea Solari 40 and viale Lombardia 65. Here we can see how the increase of importance of the city center and extension of the population forced those project at what were then the city boundaries. From now on no major public housing project will happen inside the city center except for the postwar reconstruction of bombed buildings.

Three important wartime buildings were constructed in between 1940 and 1941 by Istituto Fascista Autonomo per le Case Popolari (Autonomous Fascist Institute for Public Housing). The period after the second world war can be defined as "home for all"  period, with standardized construction practices and housing units. In the years from 1959 al 1979 Milan experienced the construction of the biggest projects in order to meet the increasing demand coming from a period of high economic growth. This projects are located in Baggio, Bruzzano, Gallaratese, Quarto Oggiaro and Quinto Romano. In the 1980s buildings were constructed in the city center to replace those that had been bombed. Until 1999 continues the development in the city outskirts. In between  2000 e il 2012 14 housing project were completed. 

Comune di Milano today owns 26'489  public housing units (Servizio
Abitativo Pubblico-SAP) around the city of Milan. Those are administrated by the public owned company MM s.p.a. Comune di Milano is not the only provider of public housing around the city, also Regione Lombardia with the public owned company ALER provides 32.017 units, ALER housing units are in reduction due to budgetary needs. MM and ALER combined provide SAP units that amount for 10\% of all housing units in the city of Milan. In comparison the national average is 4\%, on the other hand the European Union average is 20\% \parencite{comune2023}.

\begin{table}
	\centering
	\begin{tabular}{lrrr}
		\toprule
		Owner & Occupied & Free & Free because of lack of maintenance\\
		\midrule
		MM & 21737&1021&3731\\
		ALER& 24333&394&3274\\
		\bottomrule
	\end{tabular}
	\caption{Public Housing Stock by usage}
\end{table}

Demand for public housing is high. Comune di Milano administrates both the allocation of MM and ALER housing units. In 2021 11916 applications were received, in 2022 17.785 and at the end of the year 16.468. The high demand does not allow the allocation of units to all the target population of public housing policies. On average the ISEE of tenants is 5000 €, the policy target is to provide public housing for all families below 16000€. Only a small fraction of the tenants that cannot receive SAP units will receive another type of rent protection, the others will be left to choices on the free market. \parencite{comune2023}

 The main legal framwork that norms public housing in Lombardy comes from two regional laws: 16/2016 and 4/2017. Those also highlight the prerequisites for accessing public housing services, that are:
\begin{itemize}
	\item Citizenship requirement: Being an Italian or EU member state citizen or being a permanent resident
	\item Residency requirement: Being resident in Lombardy or working in the region
	\item No arrears that caused a removal from a previous assignation to public housing 
	\item ISEE below 16000€
\end{itemize} 
\parencite{Incorvaia}



FUTURE OF PUBLIC HOUSING IN MILAN 

\section{Data}\label{data}

We combine data form the Open Data portal of Comune di Milano with administrative data from MM. Comune di Milano provides us with location of services, whether MM keeps track of all public housing building maintained by the company. 

Locations of services can be extracted with many different methodologies. The most used in literature is extracting location from web based map services such as Google Maps or the Open Street Map. Those services are not necessary for this application as the Open Data portal already provides coordinates of many services. More importantly many services targeted towards the most vulnerable will be not easily extracted from the web. Consequently, The acquisition of georeferenced data for services can be replicated in two different ways: by obtaining administrative data from similar portals as the one offered by Comune di Milano or shift to web based map services.

Services included in this analysis have the objective to represent the necessities of the users of public housing. We decided to include services that fall under five categories:
\begin{itemize}
	\item Education and culture: schools, universities, libraries, public Wi-Fi, and cultural points of interests (such as museums and monuments).
	\item  Retail and similar: stores, groceries, supermarkets, pharmacies, filtered water distribution points, news-stands and postal offices. 
	\item Counselling: family and addiction counselling.
	\item Leisure: sport facilities and parks.
	\item Transports: bike lanes, trains and metro stations.
\end{itemize}
Most services have a function that is easily inferable, but some require a more detailed explanation. Water fountains are public fountains that filter and add carbon dioxide to water. They should not be mistaken for the famous "vedovelle" that are more diffused and smaller but  do not provide filtered water. Family counselling (Consultorio familiare) is a public health service provided by Regione Lombardia that provides support for all citizens (individuals, families and children) on all issues regarding relationship, family and sexual life. Patients are followed by multidisciplinary team of doctors, psychologists, paramedics, lawyers and social workers. Examples of services provided are support regarding birth control, breastfeeding, menopause, psychological issues of individuals or families, family law and female cancer prevention. Cultural points of interests comprehend museums, public art and historical monuments. Postal offices have some additional responsibility compared to other countries. They also offer  insurance and other financial products but most notably eligible citizens can withdraw pensions and state provided income support policies. Addiction counselling (Servizi per le Dipendenze patologiche or SERD) is a public service provided by the national health service that supports all individuals subjects to addiction. Like family counselling, professional form different background can support SERD patients.  Services provided are prevention, treatments, social reintegration of individuals that suffer from any type fo addiction. All services are geoencoded as data points, also parks and bike lanes, where the area covered is converted into points. This allows for a large park or a long bike lane close to a public housing building to be have more importance in subsequent analysis compared to a smaller or shorter counterpart.  


\begin{table}
	\centering
	\begin{tabular}{lrl}
		\toprule
		Dataframe Name&Number of Services&Short Description\\
		\midrule
		acqua&52& Water fountains\\
		biblio&26& Public libraries\\
		ciclabili&3722&Bike lanes (data points)\\
		consu&21&Family counselling\\
		cult&	322&Cultural POI\\
		distr&	1027&Stores and Groceries\\
		edicole&	577&News-stands\\
		farmacie&	414& Pharmacies\\
		metro&	110& Metro stations\\
		parchi&	1065& Public parks (data points)\\
		posta&	82& Postal offices\\
		serd&	15& Addiction counselling\\
		sinf&	300& Kindergarten\\ 
		sita&	115&Italian schools\\
		sport&	1041& Sport facilities\\
		sprim&	1734&Elementary schools\\
		ss2&174&High schools\\
		ssec&1146&Middle schools\\
		treni&24&Train stations\\
		uni&711&University buildings\\
		
		\bottomrule
	\end{tabular}
	\label{servtab}
	\caption{Services data composition}
\end{table}

In order to analyze the connectivity of each building of social housing with different services we need to create a matrix of distances. The distance matrix will be shaped as follows:
\[
D_{m\times n} =
\left[ {\begin{array}{cccc}
		d_{11} & d_{12} & \cdots & d_{1n}\\
		d_{21} & d_{22} & \cdots & d_{2n}\\
		\vdots & \vdots & \ddots & \vdots\\
		d_{m1} & d_{m2} & \cdots & d_{mn}\\
\end{array} } \right]
\]
With $m$ representing the total number of social housing buildings owned by Comune di Milano and $n$ representing all the services. Our distance matrix is quite big as we have 984 buildings and 13275 services. A total of $13,062,600$ distances needs to be calculated. This will indeed be a long and costly process using an API service like the ones offered by Google Maps, thus we moved to locally computable and open source alternatives. We implement the backend of The Open Source Routing Machine \parencite{luxen-vetter-2011} in our code in order to calculate the distance matrix. 

The Open Source Routing Machine (OSRM) is a cross platform and open source routing engine. It is written in C++ and is highly optimized: it can process distances between two coordinates points in milliseconds\footnote{15.7 seconds for 1000 requests with 8 GB of RAM and i7-8550U CPU @ 1.80GHz}. 

Another advantage of this process as highlighted by \cite{Huber2016} is that is a fully local method. This allows firstly to be implemented with data that is highly sensible and cannot be shared with API providers. Secondly reproducibility of research is improved by being fully detached from online services. Sometimes online services are preferable as they allow for traffic information to be included in travel time estimates, in a recent version, this feature was also added to the OSRM. 

An OSRM running on all the planet is impossible to develop without the computing capabilities of a server, but a local version (in our case limited to Provincia di Milano) is more than accessible even to commercial PCs.

We now describe the inner workings of OSRM. Routing is an example of application of shortest path problems, where the road network is modelled into a weighted and directed graph $G = (V,E)$. Junctions are represented by nodes and street segments by edges. All this information is extracted from OpenStreetMap.  An impractical solution to this problems can be derived form Dijkstra's algorithm \parencite{algo}, but this approach necessitates to analyze an immense number of edges necessary to solve a routing problem. OSRM uses a contraction hierarchy (CH) algorithm to speed up the routing problem. CH algorithms preprocesses the graph by exploiting the tendency of road network to be constructed by few important and many unimportant roads and junctions, this is known as an hierarchical structure: Local roads, for example, will be used only around the start and destination points and not along most of the route. Note that it is not necessary for the algorithm to know in advance the type or roads, but it estimates that autonomously. CH algorithms thus works by removing unimportant nodes from the calculation, by creating  shortcuts that bypass nodes but still represents total paths. The number of shortcuts in the graph is crucial as an excess of it will affect preprocessing time and resources, heuristic functions are thus implemented to keep the number of shortcuts optimal. \parencite{Geisberger2012}

\section{Empirical Design}
\label{edesign}
Indexes used in evaluation of spacial accessibility are  categorized by \cite{marwal_literature_2022} in isochrones, gravity or utility based. Isochrones provide a simple and broad view of accessibility, but count only the locations accessible when below a certain defined boundary. Gravity based measures weight opportunities using  a functional form of travel time or costs, they can be also integrated with features of locations such as quality of treatment for hospitals or similar. Utility based models rely on random utility theory to evaluate locations, knowing that agents will select the ones that maximize their utility. They require low input data but still some is necessary in order to estimate parameters of the utility function \parencite{ziemke_accessibility_2018}.

We create an index for connectivity (CI) with main objective of facility of implementation. Thus we chose a gravity based model with a functional form that rewards locations that are closer to a specific distance. Note that still all locations are considered when calculating the index. The connectivity index satisfies the following conditions :
	\begin{enumerate}
		\item $0<\text{CI}<1$
		\item CI depends on the minimum distance to the element of a given service (minimum distance or MD)
		\item CI depends also on  the total number of elements accessible in less than a threshold distance (number of elements in neighborhood or NEN). This value is relative to the building with the most elements of a given service inside the minimum threshold distance
		\item It allows for a different weighting given to MD and NEN
		\item It gives a ``penalty'' to all those services where the closest element is more distant than fifteen minutes.
	\end{enumerate}
	We define $\boldsymbol{s}_i$ as the vector containing all distances to services in the city for building $i$. We denote $d_{min} \in \boldsymbol{s}_i: d_{min}\leq d_j \forall d_j \in  \boldsymbol{s}_i$, that is the minimum distance for service $i$ (MD). We set $\bar{d}$ as the threshold for close services. $n$ will be the number of elements in the vector where $d<\bar{d}$ (NEN). $n_{max}$ will be the maximum number of close services for any building, thus representing the best in class for number of services inside the threshold distance. We need this measure as some services are more frequent around the city than others. Our index will be:
	\begin{equation}
		\text{CI}=\alpha f\left(d_{min}\right)+\beta \left(\frac{n}{n_{max}}\right)
	\end{equation}
	where:
	\begin{equation*}
		f(x)=
		\begin{cases} 
			1 & x\leq \bar{d} \\
			\frac{1}{x-\bar{d}+1} & x>\bar{d}
		\end{cases}
	\end{equation*}
 Note that $\alpha$ and $\beta$ must be chosen such that $\alpha+\beta=1$
 
 \begin{figure}
 	\centering
 \begin{tikzpicture}
	\begin{scope}[xshift=6cm]
		\begin{axis}[%
			title=Penality funciton for minimun distance,
			ylabel={$f(d_{min})$},
			xlabel={$d_{min}$ (s)},
			xticklabels={}
			]
			\draw [dashed] (axis cs:1,0) -- (axis cs:1,1);
			\node [left] at (axis cs:  1, 0.6) {$\bar{d}$};
			\addplot[smooth, thick,domain=\Tolerance:1-\Tolerance,samples=100]{f(x)};
			\addplot[smooth, thick,domain=1+\Tolerance:4,samples=100]{f(x)};
		\end{axis}
	\end{scope}
\end{tikzpicture}
 \end{figure}

\section{Results}
\label{res}

We created the distance matrix for each service and each public housing building as presented in section \ref{data}. In order to extract insights form those data we applied our connectivity index to the distance matrix. We chose to implement a parametrization of CI that assigned equal value to closeness ($\alpha$ parameter) and variability of services ($\beta$ parameter). This implies that $\alpha=\beta=0.5$. Another parameter that needs to be exogenously determined is $\bar{d}$, that is the threshold distance for close services. Political and academic debate on the "15 minute city" was the main driver of parametrization choice for threshold distance of $\bar{d}=900s$ \parencite{khavarian-garmsir_15-minute_2023}. Although this threshold is deterministic, in our index it is not the only mode of evaluation, as our index is not entirely based on this value.

Table \ref{suumstatstab} reports summary statistics for the connectivity index of each service. We aggregate the CI calculated for each building to the service that this refers to in order to have a city wide metric differentiated by type of service. This means, for example, that the average reported for train stations is the average CI across public housing buildings for the service of train stations.

From the average we see that 14 services out of 22 have average CI scores of $0.5\pm0.15$. Cultural points of interest, family and addiction counselling university buildings and train stations are the services that score on average lower than this range. Cultural points obtain an exceptionally low value of $0.058004$ on average. This values could be a consequence of the tendency of cultural points of interest to be placed in the city center and thus away from the majority of public housing buildings. This view is also confirmed by metrics described in the following paragraphs. The median provides similar insights. We see from maximum and minimum that our index ranges exactly from 1 to $0\pm0.00003$ for all services. This was indeed expected from the design of the connectivity index.

It is also interesting to see how CI scores are distributed among public housing units. For this purpose we use skewness and kurtosis. Skewness estimates the third standardized moment of the population distribution and it is an index of the symmetry of the distribution. It ranges from $-\infty$ to $+\infty$, it is $0$ when the distribution is perfectly symmetric and takes negative or positive values based on were the asymmetry lays. Kurtosis is an estimate of the fourth standardized moment of the population distribution and it is used to compare the peakedness (weaker or heavier tails and shoulders) of a distribution to the ones of the normal. It ranges from $1$ to $\infty$ and for reference the normal distribution has a kurtosis of $3$. Often to this metric is subtracted 3 (as in our case) to better highlight the relation with the normal distribution  \parencite{ho_descriptive_2015}. From table \ref{suumstatstab} we note that the most skewed distribution is the distance from Duomo with a value of $12.600432$. This implies that the distribution of the geodesic distance from the city center is asymmetric and the  right tail extends the most. This metric shows that there is a concentration of public housing units are placed close to the city center. Kurtosis is also high, implying that this metric is more has heavier tails compared to the gaussian distribution. A concentration of data has low value of distance from city center but outliers are quite frequent.  Looking at results for the CI we see that the highest value of skewness is reached by cultural points of interest. Here the interpretation differs as a higher values imply better connectivity. This service has finds most of the data on the lower scores of CI implying that most public hosing is poorly connected with cultural points of interest. Kurtosis is also high showing that there are more outliers compared to the normal. We see that the services that come closest to having a symmetric distribution are metro stations, with a value of $0.384926$. Notably most of the services obtain a negative value of skewness, implying that most of the scores have high values. Store and groceries seems to be the most left with a value of $-3.172366$. Looking at kurtosis 12 out of 22 services have values higher than zero and thus tails heavier than the gaussian distribution. Postal offices, italian schools and water fountains come very close to 0 with values of $-0.082454$, $-0.069862$ and $0.074387$ respectively. Those are the services that ave distributions most similar to the normal.

 

\begin{landscape}
	\begin{table}
		\centering
		\begin{tabular}{lrrrrrrr}
			\toprule
						  Service &        Mean &      Median &    Minimum &      Maximum &    $\sigma$ &  Skewness &   Kurtosis \\
			\midrule
				  Water fountains &    0.588909 &    0.666667 &   0.000038 &     1.000000 &    0.298981 & -1.256788 &   0.074387 \\
				 Public libraries &    0.448982 &    0.666667 &   0.000035 &     1.000000 &    0.330759 & -0.545798 &  -1.568690 \\
					   Bike lanes &    0.583804 &    0.581871 &   0.000043 &     1.000000 &    0.098156 & -2.484344 &  18.249688 \\
			   Family counselling &    0.180754 &    0.000914 &   0.000032 &     1.000000 &    0.285379 &  1.013261 &  -0.832189 \\
					 Cultural POI &    0.058004 &    0.000417 &   0.000033 &     1.000000 &    0.183163 &  3.140419 &   8.898860 \\
					  News-stands &    0.579617 &    0.568966 &   0.000039 &     1.000000 &    0.094452 & -1.154315 &  15.855984 \\
					   Pharmacies &    0.614192 &    0.612903 &   0.000040 &     1.000000 &    0.125522 & -1.438254 &  10.073702 \\
				   Metro stations &    0.384926 &    0.550000 &   0.000043 &     1.000000 &    0.305534 & -0.305707 &  -1.527592 \\
					 Public parks &    0.502103 &    0.531792 &   0.000036 &     1.000000 &    0.181710 & -1.966817 &   3.413313 \\
				   Postal offices &    0.493276 &    0.562500 &   0.000039 &     1.000000 &    0.259047 & -1.147095 &  -0.082454 \\
			Addiction counselling &    0.191438 &    0.000843 &   0.000033 &     1.000000 &    0.291208 &  0.976176 &  -0.752664 \\
					 Kindergarten &    0.683411 &    0.678571 &   0.000039 &     1.000000 &    0.115460 & -2.197506 &  12.323776 \\
				  Italian schools &    0.466289 &    0.535714 &   0.000038 &     1.000000 &    0.247952 & -1.068808 &  -0.069862 \\
				 Sport facilities &    0.693574 &    0.676724 &   0.000040 &     1.000000 &    0.122241 & -1.134635 &   8.502487 \\
			   Elementary schools &    0.708579 &    0.700000 &   0.000040 &     1.000000 &    0.134471 & -1.730198 &   8.854016 \\
					 High schools &    0.422154 &    0.541667 &   0.000035 &     1.000000 &    0.281271 & -0.668483 &  -1.149812 \\
				   Middle schools &    0.608370 &    0.593023 &   0.000039 &     1.000000 &    0.135834 & -2.820057 &  11.689541 \\
				   Train stations &    0.222444 &    0.002111 &   0.000034 &     1.000000 &    0.304051 &  0.684745 &  -1.456602 \\
			 University buildings &    0.212721 &    0.001405 &   0.000037 &     1.000000 &    0.271683 &  0.630182 &  -1.281901 \\
							 wifi &    0.616373 &    0.603774 &   0.000039 &     1.000000 &    0.100988 & -1.044054 &  10.986417 \\
			 Stores and Groceries &    0.509896 &    0.515695 &   0.000038 &     1.000000 &    0.108288 & -3.172366 &  16.078387 \\
					 All services &    0.626734 &    0.613839 &   0.000043 &     1.000000 &    0.074860 &  0.992269 &  10.011883 \\
					   duomo\_dist & 5213.334094 & 4933.363499 & 318.419822 & 61032.302044 & 2533.143374 & 12.600432 & 277.884758 \\
			\bottomrule
			\end{tabular}
			
			
			\label{suumstatstab}
			\caption{Summary statistics on the connectivity index for each service. $\sigma$ denotes standard deviation.}
		\end{table}	
		\end{landscape}

Table \ref{statstab} reports three statistics that further   evaluate the connectivity index across services. Those are the Gino coefficient, correlation with geodesic distance from city center and mutual information. 

The Gini coefficient adds a metric to estimate how equally the connectivity index is distributed among buildings. The Gini coefficient is calculated using the Lorenz curve, that is  a polygonal line connecting points that have coordinates given by the cumulative relative frequency of a certain value, and arranged increasingly. This curve is the compared with the egalitarian line, that is the Lorenz curve were all variables have the same value. The index is the ratio of the area in between the two curves and the area below the egalitarian line. The Gini coefficient ranges from 0 to1. Note that the Gini coefficient is 0 when the Lorenz curve and the egalitarian line coincide, this corresponds to perfect equality. The converse is true when the Gini index is 1 \parencite{giorgi_gini_2020}. Although the Gini coefficient is often applied to income and wealth inequality, it can provide insights also when applied to the connectivity index. In this case the value 0 will represent a perfect equality of access to a specific service among the public housing buildings. We see that the lowest Gini coefficient is achieved when considering all services across the city. This implies that overall access to services with no differentiation is quite equally distributed across public housing units. MOst services perform well with nine of the under the threshold of $0.1$. Only fro services perform more than $0.5$. Those are university buildings, train stations, addiction and family counselling and cultural points ov interest, the worst service when it comes to equality of scores. 

We add a metric for geodesic distance in meters from the city center, that we identify with the coordinates of Duomo di Milano. Python's Geopy library allows us to calculate the geodesic distance between two points using the World Geodetic System model, the most used for this type of calculations. We also provide summary statistics for this metric and we see that on average, each public housing unit is a 5968.278 meters away from the city center. The correlation column reports the Pearson correlation between each service connectivity index. We want to investigate if there is some correlation between availability of services and the distance of the building form the city center. Notably each service has negative correlation between CI and distance from Duomo. This suggests that as distance of public housing building from city center increases, the availability of all services decreases. Although the previous insight is true for all services, we see a wide range of value of correlation. For example, overall services have the most negative correlation with a value of $-0.596738$. Notably for policy implications, elementary schools, middle schools and family counselling are quite negatively correlated with distance from the city center with values around $-0.4$. Public parks, public libraries, postal offices, italian schools
and addiction counselling have correlation values of $-0.2$ and $-0.1$. Train stations and water fountains have almost no correlation with distance from city center, implying that those services are quite diffused around the city. 

Pearson's correlation investigates linear relationships between variables. A more general measure comes form mutual information that observes also monotonic and non-monotonic relationships. We replicate the same analysis as for the correlation investigating the relation between distance from the city center and connectivity index. Mutual information as described in \cite{elements} is "a measure of the amount of information that one random variable contains about another random variable". We see that results here change substantially compared to the ones showed previously. This was expected as mutual information investigates more complex types of relations between variables. Mutual is equal to 0 when no information is provided and is maximum when the two random variables are the same. This relationship is described by the following equation
\[I(X;Y)=H(X)-H(X|Y)\]
Were $H(X)$ is the Shannon's entropy of the random variable $X$ and $H(X;Y)$ the conditional entropy of $X$ and $Y$. From the previous we get that: 
\[I(X;X)=H(X)\]
Thus in our case the upper bound is $5.553744$, that is the mutual information that of the same variable (distance form Duomo). We see that the cultural points of interest come very close to this bound, although they obtained lower correlation. The same can be said about university buildings and all services. On the other hand, middle, elementary and italian schools, public libraries, postal offices, kindergarten and water fountains perform the lowest value of mutual information (less than $3.000$). 

\begin{table}
\centering
\begin{tabular}{lrrr}
	\toprule
				  Service &     Gini &  Correlation &  Mutual Information \\
	\midrule
		  Water fountains & 0.240346 &    -0.053253 &            1.913922 \\
		 Public libraries & 0.367927 &    -0.193059 &            2.533260 \\
			   Bike lanes & 0.071857 &    -0.376415 &            4.534293 \\
	   Family counselling & 0.721401 &    -0.382185 &            4.458782 \\
			 Cultural POI & 0.910358 &    -0.351438 &            5.394022 \\
			  News-stands & 0.069450 &    -0.512833 &            3.071575 \\
			   Pharmacies & 0.091634 &    -0.470427 &            2.875015 \\
		   Metro stations & 0.415849 &    -0.276610 &            3.318674 \\
			 Public parks & 0.154304 &    -0.196006 &            3.898988 \\
		   Postal offices & 0.253366 &    -0.162762 &            2.340985 \\
	Addiction counselling & 0.703958 &    -0.124862 &            4.360059 \\
			 Kindergarten & 0.083242 &    -0.375858 &            2.330676 \\
		  Italian schools & 0.255305 &    -0.140015 &            2.651912 \\
		 Sport facilities & 0.089091 &    -0.280608 &            3.641130 \\
	   Elementary schools & 0.095642 &    -0.428292 &            2.814154 \\
			 High schools & 0.342531 &    -0.304161 &            3.529942 \\
		   Middle schools & 0.092428 &    -0.395775 &            2.565271 \\
		   Train stations & 0.657674 &    -0.076874 &            4.122878 \\
	 University buildings & 0.635098 &    -0.276371 &            4.844099 \\
					Public Wi-Fi & 0.079492 &    -0.478185 &            3.303520 \\
	 Stores and Groceries & 0.065545 &    -0.338781 &            3.333871 \\
			 All services & 0.057605 &    -0.596738 &            5.038962 \\
			   Distance from Duomo & 0.188589 &     1.000000 &            5.553744 \\
	\bottomrule
	\end{tabular}
	
	\label{statstab}
	\caption{Statistics on the connectivity index for each service. Correlation column reports the Pearson correlation between the connectivity index and geodesic distance in meters from the city center (identified with Duomo di Milano). Gini column reports the Gini coefficient of each service. MI column reports the mutual information between the connectivity index and distance from city center}
\end{table}	

\begin{landscape}
	\scriptsize
	\begin{table}
	\centering
		
	

	\begin{tabular}{rrrrrrrrrrrr}
		\toprule
		\multicolumn{4}{l}{consu} & \multicolumn{4}{l}{serd} & \multicolumn{4}{l}{sinf} \\
		  Max &   ID max &      Min &   ID min &  Max &   ID max &      Min &   ID min &      Max &   ID max &      Min &   ID min \\
		\midrule
		  1.000000 & 10004301 & 0.000155 & 20002703 &  1.000000 & 30000601 & 0.000141 & 10002701 & 1.000000 & 20000804 & 0.000941 & 10002701 \\
		  1.000000 & 51015501 & 0.000155 & 20002605 &  1.000000 & 30000504 & 0.000141 & 10003901 & 1.000000 & 51010701 & 0.000951 & 10003901 \\
		  1.000000 & 10004201 & 0.000155 & 20002606 &  1.000000 & 51016301 & 0.000158 & 41010201 & 0.964286 & 20000805 & 0.002360 & 51012601 \\
		  0.800000 & 20001102 & 0.000155 & 20002607 &  1.000000 & 30003701 & 0.000187 & 10003203 & 0.964286 & 20003501 & 0.002491 & 51016001 \\
		  0.800000 & 10003001 & 0.000155 & 20002701 &  1.000000 & 30000801 & 0.000187 & 10003204 & 0.928571 & 10003801 & 0.003155 & 30001402 \\
		  0.800000 & 10004101 & 0.000155 & 20002702 &  1.000000 & 30000701 & 0.000187 & 10002101 & 0.928571 & 10003802 & 0.004664 & 51038101 \\
		  0.800000 & 20001001 & 0.000155 & 20002801 &  1.000000 & 30003402 & 0.000195 & 10002601 & 0.928571 & 51023601 & 0.005708 & 41018505 \\
		  0.800000 & 20001101 & 0.000155 & 20002604 &  1.000000 & 30003401 & 0.000195 & 10002602 & 0.928571 & 20000801 & 0.005708 & 41018504 \\
		  0.800000 & 20001103 & 0.000155 & 20002603 &  1.000000 & 30000503 & 0.000195 & 10002604 & 0.928571 & 20000803 & 0.006098 & 41023301 \\
		  0.800000 & 20000301 & 0.000155 & 20002602 &  1.000000 & 30000502 & 0.000195 & 10002605 & 0.892857 & 51017501 & 0.535714 & 20001813 \\
		
		\end{tabular}

		\begin{tabular}{rrrrrrrrrrrr}
			\toprule
			\multicolumn{4}{l}{sita} & \multicolumn{4}{l}{sprim} & \multicolumn{4}{l}{ss2} \\
				 Max &   ID max &      Min &   ID min &    Max &   ID max &      Min &   ID min &      Max &   ID max &      Min &   ID min \\
			\midrule
			1.000000 & 30003701 & 0.000159 & 51016001 & 1.000000 & 20000801 & 0.000455 & 41021801 & 1.000000 & 10004201 & 0.000327 & 10002701 \\
			1.000000 & 30000601 & 0.000164 & 51038101 & 1.000000 & 51023601 & 0.000455 & 41021802 & 1.000000 & 10004301 & 0.000329 & 10003901 \\
			1.000000 & 30000502 & 0.000182 & 51013001 & 1.000000 & 20000201 & 0.000455 & 10003301 & 0.916667 & 20000805 & 0.000386 & 41010201 \\
			1.000000 & 30000503 & 0.000307 & 10004501 & 1.000000 & 20000804 & 0.000455 & 41021901 & 0.916667 & 51013901 & 0.000408 & 41021801 \\
			1.000000 & 30000501 & 0.000490 & 41019402 & 1.000000 & 20000805 & 0.000455 & 41021902 & 0.916667 & 20000804 & 0.000408 & 10003301 \\
			1.000000 & 30000504 & 0.000490 & 41019401 & 1.000000 & 51010701 & 0.001782 & 30001402 & 0.916667 & 20000201 & 0.000408 & 41021901 \\
			0.964286 & 20000401 & 0.000519 & 51023702 & 0.987500 & 10003802 & 0.002147 & 51016001 & 0.916667 & 20000801 & 0.000408 & 41021802 \\
			0.928571 & 20000101 & 0.000519 & 51023701 & 0.987500 & 51014901 & 0.002849 & 51012601 & 0.916667 & 20000802 & 0.000408 & 41021902 \\
			0.928571 & 20000102 & 0.000519 & 51023703 & 0.987500 & 41019101 & 0.003587 & 51038101 & 0.916667 & 20000803 & 0.000448 & 41018505 \\
			0.928571 & 20000103 & 0.000526 & 51023705 & 0.987500 & 41019102 & 0.003922 & 41018801 & 0.916667 & 51010701 & 0.000448 & 41018504 \\
			
			\end{tabular}

			\begin{tabular}{rrrrrrrrrrrr}
				\toprule
				\multicolumn{4}{l}{ssec} & \multicolumn{4}{l}{distr} & \multicolumn{4}{l}{all} \\
					 Max &   ID max &      Min &   ID min &      Max &   ID max &      Min &   ID min &      Max &   ID max &      Min &   ID min \\
				\midrule
				1.000000 & 41019802 & 0.000354 & 41021801 & 1.000000 & 10004301 & 0.000547 & 41021902 & 1.000000 & 10004301 & 0.502790 & 41021901 \\
				1.000000 & 10004301 & 0.000354 & 10003301 & 0.986547 & 10004201 & 0.000547 & 41021801 & 0.980469 & 20000401 & 0.502790 & 41021902 \\
				1.000000 & 41019801 & 0.000354 & 41021802 & 0.919283 & 20000401 & 0.000547 & 41021802 & 0.974330 & 10004201 & 0.502790 & 41021802 \\
				0.906977 & 10004201 & 0.000354 & 41021901 & 0.883408 & 30000601 & 0.000547 & 41021901 & 0.952567 & 20000201 & 0.502790 & 41021801 \\
				0.906977 & 10001801 & 0.000354 & 41021902 & 0.831839 & 30003701 & 0.000547 & 10003301 & 0.938058 & 20001001 & 0.502790 & 10003301 \\
				0.872093 & 51015501 & 0.000473 & 51016001 & 0.820628 & 20001001 & 0.000573 & 10002701 & 0.921875 & 20000101 & 0.504464 & 30001402 \\
				0.825581 & 20000802 & 0.000518 & 51038101 & 0.784753 & 30000501 & 0.000577 & 10003901 & 0.920201 & 20000103 & 0.506138 & 51016001 \\
				0.825581 & 51013801 & 0.000668 & 41019602 & 0.748879 & 30000502 & 0.000743 & 30001402 & 0.914062 & 20000102 & 0.507254 & 51038101 \\
				0.825581 & 20003801 & 0.000668 & 51017101 & 0.748879 & 30000504 & 0.001018 & 51012601 & 0.904576 & 30000601 & 0.515067 & 10002701 \\
				0.825581 & 51010801 & 0.000668 & 51017103 & 0.733184 & 30000503 & 0.002304 & 30001901 & 0.901786 & 51013801 & 0.515625 & 10003901 \\
				\bottomrule
				\end{tabular}
	\caption{something}
	\label{minmax}
	\end{table}		
			
			
			
		
		
		
\end{landscape}


	

\section{Conclusion}
\label{conclusion}


\newpage
\printbibliography

\newpage

\appendix

\section{Maps}
\label{maps}

This section shows two different types of map: figure \ref{servicemap} shows all services that were extracted from the Open data portal of Comune di Milano. Figure \ref{cimap} plots the locations of the social housing buildings owned by Comune di Milano and the associated connectivity index for that specific service.

\begin{figure}[h]
	\includegraphics[width=\textwidth]{services_map.pdf}
	\caption{Map of the city of Milan showing all services used for the analysis}
	\label{servicemap}
\end{figure}

\newpage

\begin{figure}
	\includegraphics[width=0.32\textwidth]{sinf.pdf}\hfill
	\includegraphics[width=0.32\textwidth]{sprim.pdf}\hfill
	\includegraphics[width=0.32\textwidth]{ssec.pdf}
	\\[\smallskipamount]
	\includegraphics[width=0.32\textwidth]{ss2.pdf}\hfill
	\includegraphics[width=0.32\textwidth]{sita.pdf}\hfill
	\includegraphics[width=0.32\textwidth]{cult.pdf}
	\\[\smallskipamount]
	\includegraphics[width=0.32\textwidth]{metro.pdf}\hfill
	\includegraphics[width=0.32\textwidth]{serd.pdf}\hfill
	\includegraphics[width=0.32\textwidth]{consu.pdf}
	\\[\smallskipamount]
	\includegraphics[width=0.32\textwidth]{distr.pdf}\hfill
	\includegraphics[width=0.32\textwidth]{parchi.pdf}\hfill
	\includegraphics[width=0.32\textwidth]{all.pdf}
	\caption{Plots of connectivity index for different services. Each dot corresponds to one  public housing building}
	\label{cimap}
\end{figure}







\end{document}